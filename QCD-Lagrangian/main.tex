% 期間 4days
\documentclass[a4paper,12pt]{article}

% --- 基本パッケージ ---
\usepackage{amssymb}
\usepackage{mathtools}
\usepackage{geometry}
\usepackage{setspace}

% --- 単位と参照のパッケージ ---
\usepackage{hyperref}
\usepackage[capitalise,noabbrev]{cleveref}

% --- ページ設定 ---
\geometry{margin=25mm}
\setstretch{1.2}

\title{QCDのラグランジアン密度の導出}
\author{しん}
\date{\today}

\begin{document}
  \maketitle
  \section{はじめに}
  本稿の目的は格子QCDを実装するために、学んでいくものの一つであるQCDのラグランジアンを定義することである。 そのために、まずは自由なディラック場のラグランジアン密度を導出し、次にゲージ対称性を導入することでQCDのラグランジアン密度を得る。

  \section{Lie群とSU(3)ゲージ理論}
  まず、QCDは強い相互作用を記述する理論である。強い相互作用はクォークとグルーオンの相互作用を記述する。クォークはフェルミ粒子であり、スピン1/2を持つ。したがって、クォーク場はディラック場として記述される。\\
  次に、クォークは3つのカラー自由度を持つ。これらのカラー自由度はSU(3)群によって記述される。したがって、QCDはSU(3)のゲージ理論である。\\
  以上より、QCDはSU(3)のゲージ理論であり、クォーク場はディラック場として記述されることが示された。

  ここで$\lambda_a (a=1,2,\dots,8)$をSU(3)の生成子とし、$T_a = \frac{\lambda_a}{2}$と定義する。
  $\lambda$はGell-Mann行列と呼ばれる。
  また、$\mathfrak{su}(3)$を回転を加え、生成子$T_a$の線形結合で表されるリー代数とする。
  すると、任意の$U \in SU(3)$は以下のように表される。
  \[
  U = e^{i\theta_a T_a}
  \]
  ここで、$\theta_a$は実数である。
  \section{ディラック場のラグランジアン密度}
  ディラック方程式は以下のように与えられる。
  \[
  (i\gamma^\mu \partial_\mu - m)\psi = 0
  \]
  ここで、$\psi$はディラックスピノル、$m$は質量、$\gamma^\mu$はディラック行列である。\\
  この時、第一項が運動項、第二項が質量項である。\\
  これをもとに、ラグランジアン密度を以下のように仮定する。
  \[
  \mathcal{L} = i\gamma^\mu \partial_\mu \psi - m \psi
  \]
  しかし、ラグランジアンはスカラーを返す関数である必要があるため、4次スピノルを返すこの関数は不適である。
  ここで、このスピノルをスカラーに直すために、共役スピノル$\bar{\psi}$を導入する必要がある。
  他にも$\bar{\psi}$は以下の特性を満たす必要がある
  \begin{itemize}
    \item ローレンツ変換に対して不変であること
    \item エルミート共役を取ったときに、ラグランジアン密度が実数になること
  \end{itemize}
  これらの条件を満たすものとして、以下のように定義される。
  \[
  \bar{\psi} = \psi^\dagger \gamma^0
  \]

  これを用いて、ラグランジアン密度は以下のように修正される。
  \[
  \mathcal{L} = \bar{\psi}(i\gamma^\mu \partial_\mu - m)\psi
  \]
  これが自由なディラック場のラグランジアン密度である。

  \section{QCDのラグランジアン密度}
  次に、QCDのラグランジアン密度を導出する。
  QCDのラグランジアンに要請したい条件は以下の通りである。
  \begin{itemize}
    \item ローレンツ不変であること
    \item ゲージ対称性を持つこと
    \item 自由なディラック場のラグランジアン密度を含むこと
  \end{itemize}

  \subsection{局所ゲージ対称性の導入}
  前節によって、QCDはSU(3)のゲージ理論であることが示された。
  したがって、QCDのラグランジアン密度はSU(3)の局所ゲージ対称性を持つ必要がある。
  まず、ディラック場のラグランジアン密度において、ディラック場$\psi(x)$は以下のように変換される。
  \[
  \begin{aligned}
    \psi(x) \to \psi^\prime(x) &= SU(3)\psi(x) \\
    &= e^{i\theta_a(x) T_a}\psi(x)\\
    &= \exp\left(i \sum_{a=1}^8 \theta_a(x) T_a \right)\psi(x)\\
    &= \exp\left(i \sum_{a=1}^8 \theta_a(x) \frac{\lambda_a}{2} \right)\psi(x)\\
    &\equiv U(x)\psi(x)
  \end{aligned}
  \]
  補助的に$\bar{\psi}U(x)$を計算する。
  \[
  \begin{aligned}
    \bar{\psi}^\prime &= (\psi^\prime)^\dagger \gamma^0 \\
    &= (U(x)\psi)^\dagger \gamma^0 \\
    &= \psi^\dagger U^\dagger (x) \gamma^0 \\
    &= \bar{\psi} U^\dagger (x) 
  \end{aligned}
  \]
  ここで、$\theta_a(x)$は$C^\infty$級の実関数である。
  この変換は局所ゲージ変換と呼ばれる。
  この変換に対して、ラグランジアン密度が不変である必要があるため、実際に変換を適用してみる。
  \[
  \mathcal{L} \to \mathcal{L}^\prime = \bar{\psi}^\prime(i\gamma^\mu \partial_\mu - m)\psi^\prime
  \]
  ここで、一度展開し、
  \[
  \bar{\psi}^\prime i\gamma^\mu \partial_\mu \psi^\prime - m \bar{\psi}^\prime \psi^\prime
  \]
  の各項を計算する。
  まず、運動項を計算する。
  \[
  \begin{aligned}
    \bar{\psi}^\prime i\gamma^\mu \partial_\mu \psi^\prime 
    &= \bar{\psi} U^\dagger (x) i\gamma^\mu \partial_\mu (U(x)\psi) \\
    &= \bar{\psi} U^\dagger (x) i\gamma^\mu [(\partial_\mu U(x))\psi + U(x)\partial_\mu \psi] \\
    &= \bar{\psi} U^\dagger (x) i\gamma^\mu (\partial_\mu U(x))\psi + \bar{\psi} U^\dagger (x) U(x) i\gamma^\mu \partial_\mu \psi \\
    &= \bar{\psi} U^\dagger (x) i\gamma^\mu (\partial_\mu U(x))\psi + \bar{\psi} i\gamma^\mu \partial_\mu \psi
  \end{aligned}
  \]
  次に、質量項を計算する。
  \[
  \begin{aligned}
    - m \bar{\psi}^\prime \psi^\prime 
    &= - m \bar{\psi} U^\dagger (x) U(x) \psi \\
    &= - m \bar{\psi} \psi
  \end{aligned}
  \]
  以上より、ラグランジアン密度の変換は以下のようになる。
  \[
  \begin{aligned}
    \mathcal{L}^\prime 
    &= \bar{\psi} U^\dagger (x) i\gamma^\mu (\partial_\mu U(x))\psi + \bar{\psi} i\gamma^\mu \partial_\mu \psi - m \bar{\psi} \psi \\
    &= \bar{\psi} U^\dagger (x) i\gamma^\mu (\partial_\mu U(x))\psi + \left( \bar{\psi} (i\gamma^\mu \partial_\mu - m) \psi \right) \\
    &= \bar{\psi} U^\dagger (x) i\gamma^\mu (\partial_\mu U(x))\psi + \mathcal{L}
  \end{aligned}
  \]
  ここで、第二項は元のラグランジアン密度である。
  よって、ラグランジアン密度は局所ゲージ変換に対して不変ではないことが分かる。
  したがって、ラグランジアン密度を修正する必要がある。
\subsection{修正されたラグランジアン密度}
  前節で見たように、単純な微分を用いたラグランジアン密度をゲージ変換すると、以下の「余分な項(ゴミ)」が現れてしまった。
  \[
  \delta \mathcal{L} = \bar{\psi} U^\dagger (x) i\gamma^\mu (\partial_\mu U(x))\psi
  \]
  この項を打ち消してゲージ不変性を回復させるために、ラグランジアン密度に新たな相互作用項を加えることを考える。
  
  まず、通常の微分 $\partial_\mu$ を、未知の補正項 $(\cdot)$ を含んだ共変微分 $D_\mu$ に置き換える。
  \[
  D_\mu = \partial_\mu + (\cdot)
  \]
  この $(\cdot)$ は、変換によって現れるゴミを相殺するためのものである。
  この $(\cdot)$ が $3\times3$ 行列として振る舞う必要があるため、Lie代数の基底 $T_a$ と、新しい場 $A_\mu^a(x)$、結合定数 $g$ を用いて以下のように定義してみる。
  \[
  D_\mu = \partial_\mu - ig A_\mu^a T_a \equiv \partial_\mu - ig A_\mu
  \]
  この新しい共変微分を用いたラグランジアン密度を $\mathcal{L}_{\text{new}}$ とする。
  \[
  \mathcal{L}_{\text{new}} = \bar{\psi} (i\gamma^\mu D_\mu - m) \psi = \bar{\psi} (i\gamma^\mu \partial_\mu - m) \psi + g \bar{\psi} \gamma^\mu A_\mu \psi
  \]
  ここで、第二項 $g \bar{\psi} \gamma^\mu A_\mu \psi$ が、ゴミを消すために追加した新しい項である。

  では、この $\mathcal{L}_{\text{new}}$ がゲージ変換に対して不変になる条件(ゴミがきれいに消える条件)を求める。
  ゲージ変換を行い、項を展開する。
  \[
  \begin{aligned}
    \mathcal{L}_{\text{new}}^\prime &= \bar{\psi}^\prime (i\gamma^\mu \partial_\mu - m) \psi^\prime + g \bar{\psi}^\prime \gamma^\mu A_\mu^\prime \psi^\prime \\
    &= (\mathcal{L}_{\text{orig}} + \underbrace{\bar{\psi} U^\dagger i\gamma^\mu (\partial_\mu U) \psi}_{\text{元のゴミ}}) + g (\bar{\psi} U^\dagger) \gamma^\mu A_\mu^\prime (U \psi) \\
    &= \mathcal{L}_{\text{orig}} + \bar{\psi} i\gamma^\mu \left( U^\dagger (\partial_\mu U) - ig U^\dagger A_\mu^\prime U \right) \psi
  \end{aligned}
  \]
  この式が、変換前の形($\mathcal{L}_{\text{orig}} + g \bar{\psi} \gamma^\mu A_\mu \psi$)に戻るためには、括弧の中身が $-ig A_\mu$ になればよい。
  すなわち、以下の等式が成立する必要がある。
  \[
  U^\dagger (\partial_\mu U) - ig U^\dagger A_\mu^\prime U = -ig A_\mu
  \]
  この式を、未知の場 $A_\mu^\prime$ について解く。
  全体に左から $U$ を掛け、符号を整理すると、
  \[
  \begin{aligned}
    (\partial_\mu U) - ig A_\mu^\prime U &= -ig U A_\mu \\
    - ig A_\mu^\prime U &= -ig U A_\mu - (\partial_\mu U) \\
    A_\mu^\prime U &= U A_\mu + \frac{1}{ig} (\partial_\mu U) \\
    A_\mu^\prime &= U A_\mu U^\dagger - \frac{i}{g} (\partial_\mu U) U^\dagger
  \end{aligned}
  \]
  この結果は以下のことを意味する。
  「もし、新しく導入した場 $A_\mu$ が、上記のルールに従って変換してくれるならば、ラグランジアンから発生するゴミは完全に相殺され、ゲージ不変性が保たれる」
  \[
  \begin{aligned}
    A_\mu \to A_\mu^\prime = U A_\mu U^\dagger - \frac{i}{g} (\partial_\mu U) U^\dagger\\
  \Rightarrow \mathcal{L}_{\text{new}} \to \mathcal{L}_{\text{new}}^\prime = \mathcal{L}_{\text{new}}
  \end{aligned}
  \]
  以上より、QuarkとGluonの相互作用を含むQCDのラグランジアン密度は以下のように与えられる。
  \[
  \mathcal{L}_{\text{QCD}} = \bar{\psi} (i\gamma^\mu D_\mu - m) \psi
  \]
  ここで、共変微分は以下のように定義される。
  \[
  D_\mu = \partial_\mu - ig A_\mu^a T_a
  \]
  これはつまり
  \[
  D_\mu = \partial_\mu - ig A_\mu^a \frac{\lambda_a}{2}
  \]
  である。

  さらに、ラグランジアンがSU(3)の局所ゲージ対称性を持つためには、Gluon場自身の運動項も必要である。
  そのために、$D_\mu\psi(x)$が
  \[
  D_\mu\psi(x) \to D_\mu^\prime \psi^\prime(x) = U(x) D_\mu \psi(x)
  \]
  というゲージ変換を満たす必要がある。
  これを満たすために、共変微分同士の交換子を考える。
  \[
    D_\mu^\prime \psi^\prime(x) = \left[\partial_\mu - ig \frac{\lambda_a}{2} A_\mu^{a \prime}(x)\right] U(x) \psi(x) 
  \]
  せっかくなので共変性を活かして(展開してゴミを消して...)書き直すと、
  \[
   D_\mu^\prime \psi^\prime(x) = U(x) \left[\partial_\mu - ig \frac{\lambda_a}{2} A_\mu^{a}(x)\right] \psi(x)
  \]
  となる。
  これは簡単に
  \[
  D_\mu^\prime \psi^\prime(x) = U(x) D_\mu \psi(x)
  \]
  であり、以下のような等式を満たす必要がある。
  \[
  \left[\partial_\mu - ig \frac{\lambda_a}{2} A_\mu^{a \prime}(x)\right] U(x) \psi(x) = U(x) \left[\partial_\mu - ig \frac{\lambda_a}{2} A_\mu^{a}(x)\right] \psi(x)
  \]
  これを展開すると、
  右辺は
  \[
  U(x) \partial_\mu \psi(x) - ig U(x) \frac{\lambda_a}{2} A_\mu^{a}(x) \psi(x)
  \]
  となり、左辺は
  \[
  \partial_\mu U(x) \psi(x) + U(x) \partial_\mu \psi(x) - ig \frac{\lambda_a}{2} A_\mu^{a \prime}(x) U(x) \psi(x)
  \]
  となる。
  これらを等式に代入し、整理すると、
  \[
  \partial_\mu U(x) \psi(x) - ig \frac{\lambda_a}{2} A_\mu^{a \prime}(x) U(x) \psi(x) = - ig U(x) \frac{\lambda_a}{2} A_\mu^{a}(x) \psi(x)
  \]
  となる。
  ここで、両辺を $\psi(x)$ で割り、$\frac{\lambda_a}{2}$ でくくると、
  \[
  - ig A_\mu^{a \prime}(x) U(x) = - ig U(x) A_\mu^{a}(x) - \partial_\mu U(x)
  \]
  となる。
  これを $A_\mu^{a \prime}(x)$ について解くと、
  \[
  A_\mu^{a \prime}(x) = U(x) A_\mu^{a}(x) U^\dagger(x) - \frac{i}{g} (\partial_\mu U(x)) U^\dagger(x)
  \]
  となり、前節で求めた変換則
  \[
  A_\mu^\prime = U A_\mu U^\dagger - \frac{i}{g} (\partial_\mu U) U^\dagger
  \]
  と一致する。
  したがって、共変微分は
  \[
  D_\mu = \partial_\mu - ig A_\mu^a T_a
  \]
  され、ここで$D_\mu$の変換則は
  \[
  D_\mu^\prime= U(x) D_\mu U^\dagger(x)
  \]
  である。($\psi$を省いた形)
  ここで、$A_\mu^a$ はGluon場であり、$T_a$ はSU(3)の生成子である。
  よって、ここまでのラグランジアンは以下のように記述される。
  \[
  \mathcal{L}_{\text{QCD}} = \bar{\psi}\left[i\gamma^\mu (D_\mu\psi) - m\psi\right]
  \]
  しかし、これはQuarkとGluonの相互作用のみを記述しており、Gluon場自身の運動項が含まれていない。
  そこで、次節でGluon場の運動項を導入する。

  \subsection{Gluon場の運動項}
  Gluon場自身の運動項を導入するためには、ゲージ不変な量を構成する必要がある。
  そのために、共変微分の交換子を考える。
  これは、$x \to \mu \to \nu \to x$ の閉じた無限小のループ(Lie群!)を表す。
  具体的には、
  \[
  [D_\mu, D_\nu] \psi(x)
  \]
  を計算する。まず、共変微分
  \[
  D_\mu = \partial_\mu - ig A_\mu
  \]
  を用いると、
  \[
  \begin{aligned}
  [D_\mu, D_\nu]
  &= [\partial_\mu - ig A_\mu,\ \partial_\nu - ig A_\nu] \\
  &= (\partial_\mu - ig A_\mu)(\partial_\nu - ig A_\nu) - (\partial_\nu - ig A_\nu)(\partial_\mu - ig A_\mu) \\
  &= \partial_\mu \partial_\nu - ig \partial_\mu A_\nu - ig A_\mu \partial_\nu + g^2 A_\mu A_\nu \\
  &\quad - \partial_\nu \partial_\mu + ig \partial_\nu A_\mu + ig A_\nu \partial_\mu - g^2 A_\nu A_\mu \\
  &= -ig (\partial_\mu A_\nu - \partial_\nu A_\mu) - ig (A_\mu \partial_\nu - A_\nu \partial_\mu) + g^2 (A_\mu A_\nu - A_\nu A_\mu) \\
  &= -ig(\partial_\mu A_\nu - \partial_\nu A_\mu) - g^2 [A_\mu, A_\nu]
  \end{aligned}
  \]
  となる。
  ここで、$A_\mu$ はLie代数の元(生成子の線形結合)であるため、
  \[
  A_\mu = A_\mu^a T_a
  \]
  と展開できる。
  ここで、SU(3)生成子の交換関係
  \[
  [T_a, T_b] = i f^{abc} T_c
  \]
  を考える。ここで、$f^{abc}$ はSU(3)の構造定数である。
  構造定数については、付録で詳しく説明する。
  これを用いると、
  \[
  [A_\mu, A_\nu]
  = A_\mu^a A_\nu^b [T_a, T_b]
  = i f^{abc} A_\mu^a A_\nu^b T_c
  \]
  となる。

  以上より、交換子は
  \[
  \begin{aligned}
  [D_\mu, D_\nu]
  &= -ig (\partial_\mu A_\nu^a - \partial_\nu A_\mu^a) T_a - g^2 (i f^{abc} A_\mu^b A_\nu^c T_a) \\
  &= -ig \left[ (\partial_\mu A_\nu^a - \partial_\nu A_\mu^a) + g f^{abc} A_\mu^b A_\nu^c \right] T_a \\
  \end{aligned}
  \]
  と書け、ここで$[\cdot]$を$F_{\mu\nu}^a$と定義すると、
  \[
  [D_\mu, D_\nu] = -ig F_{\mu\nu}^a T_a
  \]
  と表せる。
  ここで、$F_{\mu\nu} = F_{\mu\nu}^a T_a$ とおくと、
  \[
  [D_\mu, D_\nu] = -ig F_{\mu\nu}
  \]
  となる。
  ここで、$D_\mu$ の変換則
  \[
  D_\mu^\prime = U(x) D_\mu U^\dagger(x)
  \]
  を用いると、交換子の変換則は
  \[
  F_{\mu\nu} \to F_{\mu\nu}^\prime = U F_{\mu\nu} U^\dagger
  \]
  と変換するため、、$F_{\mu\nu}$ はゲージ変換に対して共変であることが分かる。
  したがって、$F_{\mu\nu}$ を用いてゲージ不変量を構成することができる。
  この時、matrix2scalarにしたいため、操作として$\mathrm{det}$、$\mathrm{Tr}$などが考えられる。
  しかし、$SU(3)$であるため、Supecialの性質として、$\mathrm{det}(F_{\mu\nu}) = 1$ となり、スカラー量を得ることができない。
  そこで、$\mathrm{Tr}$を用いる。
  \[
  \begin{aligned}
    \text{Tr}(F_{\mu\nu}^\prime F^{\prime\mu\nu}) 
    &= \text{Tr}(U F_{\mu\nu} U^\dagger U F^{\mu\nu} U^\dagger) \\
    &= \text{Tr}(U F_{\mu\nu} F^{\mu\nu} U^\dagger) \\
    &= \text{Tr}(F_{\mu\nu} F^{\mu\nu} U^\dagger U) \\
    &= \text{Tr}(F_{\mu\nu} F^{\mu\nu})
  \end{aligned}
  \]
  となる。
  この時、単純に$\mathrm{Tr}(F^2)$を示しているが、アインシュタインの縮約を用いているため、$\mathrm{Tr}(F_{\mu\nu} F^{\mu\nu})$と示している。
  ここで、トレースの巡回性(付録参照)を用いた。
  さらに、トレースの中身を行列から成分表示 $F_{\mu\nu} = F_{\mu\nu}^a T_a$ に戻して計算すると、
  \[
    \mathrm{Tr}(F_{\mu\nu} F^{\mu\nu}) = F_{\mu\nu}^a F^{\mu\nu b} \mathrm{Tr}(T_a T_b)
  \]
  となる。生成子 $T_a = \lambda_a/2$ の定義と、Gell-Mann行列の性質 $\mathrm{Tr}(\lambda_a \lambda_b) = 2 \delta_{ab}$ (付録参照)を用いると、
  \[
  \begin{aligned}
    \mathrm{Tr}(T_a T_b) &= \mathrm{Tr}\left(\frac{\lambda_a}{2} \frac{\lambda_b}{2}\right) 
    = \frac{1}{4} \mathrm{Tr}(\lambda_a \lambda_b) \\
    &= \frac{1}{4} \cdot 2 \delta_{ab} 
    = \frac{1}{2} \delta_{ab}
  \end{aligned}
  \]
  である。したがって、
  \[
  \mathrm{Tr}(F_{\mu\nu} F^{\mu\nu}) = \frac{1}{2} F_{\mu\nu}^a F^{a\mu\nu}
  \]
  が得られる。電磁気学($U(1)$ ゲージ理論)との整合性から、成分表示におけるグルーオンの運動項の係数は $-1/4$ であることが望ましい。
  以上の結果から、QCDのゲージ場のラグランジアン密度は以下のように定義される。
  \[
    \mathcal{L}_{\text{gauge}} = - \frac{1}{2}\mathrm{Tr}(F_{\mu\nu} F^{\mu\nu}) = - \frac{1}{4}F_{\mu\nu}^a F^{a\mu\nu}
  \]
  また、ここで$F_{\mu\nu}^a$ は場の強度テンソルであり、以下のように定義される。
  \[
  F_{\mu\nu}^a = \partial_\mu A_\nu^a - \partial_\nu A_\mu^a + g f^{abc} A_\mu^b A_\nu^c
  \]
  これにクォークの項を加え、連続理論におけるQCDのフルラグランジアン密度が完成する。
  \[
  \mathcal{L}_{\text{QCD}} = \bar{\psi}(i\gamma^\mu D_\mu - m)\psi - \frac{1}{2}\mathrm{Tr}(F_{\mu\nu} F^{\mu\nu})
  \]

  \section{結論}
  以上により、クォーク場とグルーオン場、およびそれらの相互作用を記述するQCDの完全なラグランジアン密度が導出された。
  \[
  \mathcal{L}_{\text{QCD}} = \bar{\psi} (i\gamma^\mu D_\mu - m) \psi - \frac{1}{4} F_{\mu\nu}^a F^{a \mu\nu}
  \]
  ここで、
  \begin{itemize}
    \item $D_\mu = \partial_\mu - ig A_\mu^a T_a$
    \item $F_{\mu\nu}^a = \partial_\mu A_\nu^a - \partial_\nu A_\mu^a + g f^{abc} A_\mu^b A_\nu^c$
  \end{itemize}
  である。第一項はクォークの運動とグルーオンとの相互作用を、第二項はグルーオンの運動と自己相互作用(3点および4点相互作用)を表している。

  \section{付録}
  \subsection{SU(3)の構造定数}
    構造定数とは、Lie代数の生成子の交換関係を定義する実数のテンソルである。
  直感的に言うのであれば、構造定数はLie群の「ねじれ」や「曲がり具合」を表す量である。
  簡単に$SU(2)$で、$a=1,\quad b=2, \quad c=3$の場合を考えてみる。
  \[
  [T_1, T_2] = i f^{123} T_3
  \]
  ここで、$T_i = \frac{\sigma_i}{2}$ ($\sigma_i$はパウリ行列)であるため、
  \[
  \begin{aligned}
  [T_1, T_2] &= \left[ \frac{1}{2}\begin{pmatrix} 0 & 1 \\ 1 & 0 \end{pmatrix}, \frac{1}{2}\begin{pmatrix} 0 & -i \\ i & 0 \end{pmatrix} \right]\\
  &= \frac{1}{4} \begin{pmatrix} i & 0 \\ 0 & -i \end{pmatrix} - \frac{1}{4} \begin{pmatrix} -i & 0 \\ 0 & i \end{pmatrix}\\
  &= \frac{i}{2} \begin{pmatrix} 1 & 0 \\ 0 & -1 \end{pmatrix} \\
  &= iT_3 = i f^{123} T_3, \quad f^{123} = 1
  \end{aligned}
  \]
  となる。
\end{document}