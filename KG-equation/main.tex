%期間 1day
\documentclass[a4paper,12pt]{article}

% --- 基本パッケージ ---
\usepackage{amssymb}
\usepackage{mathtools}
\usepackage{geometry}
\usepackage{setspace}

% --- 単位と参照のパッケージ ---
\usepackage{hyperref}
\usepackage[capitalise,noabbrev]{cleveref}

% --- ページ設定 ---
\geometry{margin=25mm}
\setstretch{1.2}

\title{シュレディンガー方程式の導出とクライン=ゴルドン方程式への拡張}
\author{しん}
\date{\today}

\begin{document}
  \maketitle

  \section*{はじめに}
  本稿の目的は、格子QCDを実装するために学ぶ過程として、シュレディンガー方程式を相対論的に拡張したクライン・ゴルドン方程式およびディラック方程式を導出することである。  

  まずはシュレディンガー方程式を導出し、その後クライン・ゴルドン方程式を導出する。ディラック方程式の導出は本稿では扱わない。  

  本稿では以下の条件で進める : 
  \begin{itemize}
      \item 自然単位系 $\hbar = c = 1$ は用いない。
      \item 物理的直感(ド・ブロイ波など)や量子力学の基本公理は使わず、Stoneの定理を用いて厳密に導出する。
      \item シュレディンガー方程式の導出には古典力学のハミルトニアンを用いる。
      \item クライン・ゴルドン方程式の導出には相対論的ハミルトニアンを用いる。
  \end{itemize}

  直感的には $\psi$ の関数を仮定する方法が分かりやすいが、数学的定義の方が厳密である。

  \section{導出の直感}
  今回の要請は「量子状態の時間発展が連続であること」である。

  \begin{enumerate}
    \item 量子状態空間を複素ヒルベルト空間 $\mathcal{H}$ とする。
    \item $\Psi(t)$ を量子状態とする。
    \item 量子状態の時間発展を表す演算子 $U(\epsilon)$ を定義する。
    \item $\Psi(t)$ の時間微分の責務を $U(\epsilon)$ に任せる。
    \item $U(\epsilon)$ が満たすべき物理的制約を考える。
    \item Stoneの定理を用いて $U(\epsilon)$ を自己共役演算子 $A$ を用いて表現する。
    \item $U(\epsilon)$ の微小変化を考え、$\Psi(t)$ の時間発展を微分方程式として表す。
    \item $A$をお好みのハミルトニアン$H$に置き換える。
    \item 古典ハミルトニアンの場合はシュレディンガー方程式を得る。
    \item 相対論的ハミルトニアンの場合はクライン=ゴルドン方程式を得る。
  \end{enumerate}

  \section{雛形となる微分方程式の導出}

  \subsection{初めに}
  量子状態空間を複素ヒルベルト空間 $\mathcal{H}$ とする。$\Psi(t)$ は $\mathcal{H}$ の元であり、
  \[
  \Psi(t) \in \mathcal{H}.
  \]
  時間発展演算子 $U$ を
  \[
  U(\epsilon): \mathcal{H} \rightarrow \mathcal{H}
  \]
  とする。量子状態の時間発展は
  \[
  \Psi(t + \epsilon) = U(\epsilon)\Psi(t)
  \]
  で表される。

  \subsection{時間発展演算子の性質}
  memo: $U(\epsilon)$ は物理的制約により以下を満たす
  \begin{enumerate}
    \item 線形性 : 時間が非線形に変化することはない。
    \item ユニタリ性 : 全確率を正規化すると $\|\Psi\|^2 = 1$。
    \item 1パラメータ群 : 時間発展は連続的であり、$U(t_0)U(t_1) = U(t_0+t_1)$。
  \end{enumerate}
  これがいわゆる強連続な1パラメータユニタリ群である。


  さらに $U(0)$ は単位作用素 $I$ に等しい : 
  \[
  U(0) = I \ \text{e.g.,} \begin{pmatrix} 1 & 0 \\ 0 & 1 \end{pmatrix}.
  \]

  Stoneの定理により、強連続な1パラメータユニタリ群 $U(\epsilon)$ に対して、自己共役演算子 $A$ が存在し、以下のように記される。
  \[
  U(\epsilon) = e^{-i\epsilon A}.
  \]
  この時、$A$は時間発展の生成子と呼ばれる。\\
  memo: Stoneの定理とは、強連続な1パラメータユニタリ群は必ず自己共役演算子を生成子として持つ、という定理である。\\
  memo: $A$は自己共役演算子であるため、以下の性質を満たす
  \begin{enumerate}
    \item $A = A^\dagger$ (自己共役)
    \item $A$の固有値は実数
    \item $A$の固有ベクトルは直交する
  \end{enumerate}
  テイラー展開すると微小変化は
  \[
  U(\epsilon) = I - i A \epsilon + O(\epsilon^2)
  \]
  と簡略化できる。\\

  \subsection{微分方程式の導出}
  量子状態の時間発展を表す式
  \[
  \begin{aligned}
    \Psi(t + \epsilon) &= U(\epsilon)\Psi(t)\\
    &= \big(I - i A \epsilon + O(\epsilon^2)\big) \Psi(t)
  \end{aligned}
  \]
  の時間微分について考えたいが、時間に対する微分は $\Psi(t)$ に対して行うべきであり、$U(\epsilon)$ に対して行うべきではない。
  そこで、$\Psi(t)$ の時間微分を $U(\epsilon)$ に任せることにする。すなわち、
  \[
  \begin{aligned}
    \frac{\Psi(t + \epsilon) - \Psi(t)}{\epsilon} &= \frac{U(\epsilon) - I}{\epsilon} \Psi(t) - \frac{O(\epsilon^2)}{\epsilon} \Psi(t)\\
  \end{aligned}
  \]
  が得られる。ここで $A$ とハミルトニアン $H$ は
  \[
  A = \frac{H}{\hbar}
  \]
  により対応し、整理すると
  \[
  i \hbar \frac{d}{dt} \Psi(t) = H \Psi(t)
  \]
  となる。

  \section{正準交換関係と演算子の表現}
  古典力学における位置 $x$ と運動量 $p$ は、量子力学においては演算子 $\hat{x}, \hat{p}$ となり、ハイゼンベルクの正準交換関係 (Canonical Commutation Relation, CCR)
  \[
  [\hat{x}, \hat{p}] = \hat{x}\hat{p} - \hat{p}\hat{x} = i\hbar I
  \]
  を満たすことが要請される($I$ は恒等作用素)。

  ここで、この交換関係を満たす演算子の具体的な表現形式が必要となる。\\
  ストーン・フォン・ノイマンの定理 (Stone-von Neumann Theorem) によれば、有限次元の正準交換関係(厳密にはその指数形式であるワイルの関係式)を満たす既約なユニタリ表現は、ユニタリ同値を除いて一意に定まる。

  この定理により、我々は最も一般的な「シュレディンガー表現」を採用することができる。
  位置演算子 $\hat{x}$ を座標空間における掛け算演算子として対角化する表現をとると、
  \[
  \hat{x} \psi(x) = x \psi(x)
  \]
  となる。このとき、交換関係 $[\hat{x}, \hat{p}] = i\hbar$ を満たす運動量演算子 $\hat{p}$ の表現は、微分演算子として一意に定まる :
  \[
  \hat{p} = -i\hbar \frac{\partial}{\partial x}.
  \]
  
  \begin{itemize}
      \item \textbf{証明の概略} : 波動関数 $\psi(x)$ に対して $(\hat{x}\hat{p} - \hat{p}\hat{x})\psi = i\hbar \psi$ を満たす $\hat{p}$ を $\hat{p} = A \frac{\partial}{\partial x} + B$ の形で探索すると、係数が定まる。
  \end{itemize}

  これにより、ハミルトニアン $H$ を演算子化する準備が整った。

  \section{シュレディンガー方程式の導出}
  古典力学ハミルトニアン $H$ は全エネルギー $E$ として
  \[
  H = K + V = \frac{p^2}{2m} + V(x)
  \]
  を持つ。運動量演算子は
  \[
  \hat{p} = - i \hbar \frac{\partial}{\partial x},
  \]
  よって
  \[
  \hat{H} = \frac{\hat{p}^2}{2m} + V(x) = -\frac{\hbar^2}{2m} \frac{\partial^2}{\partial x^2} + V(x)
  \]
  となる。これを微分方程式に代入すると
  \[
  i \hbar \frac{\partial}{\partial t} \Psi(x,t) = \Big(-\frac{\hbar^2}{2m} \frac{\partial^2}{\partial x^2} + V(x)\Big) \Psi(x,t)
  \]
  となる。これがシュレディンガー方程式である。

  \section{クライン=ゴルドン方程式の導出}
  相対論的ハミルトニアン $H$ は全エネルギー $E$ として
  \[
  H = \sqrt{p^2 c^2 + m^2 c^4}
  \]
  を持つ。運動量演算子は
  \[
  \hat{p} = - i \hbar \frac{\partial}{\partial x},
  \]
  よって
  \[
  \hat{H} = \sqrt{\hat{p}^2 c^2 + m^2 c^4} = \sqrt{- \hbar^2 c^2 \frac{\partial^2}{\partial x^2} + m^2 c^4}
  \]
  となる。
  \begin{quote}
    \item 根号付き演算子の定義について\\
    \ \ \ 根号付き演算子 $\sqrt{A}$ は定義が難しい。一般に、自己共役演算子 $A$ に対して、スペクトル分解を用いて定義されるが、ここでは詳細を省略する。\\
    また、定義が複雑なため、一般には付録に示している $E^2 = p^2c^2 + m^2c^4$ を直接置換する方法が一般的である。
  \end{quote}
  これを微分方程式に代入すると
  \[
  i \hbar \frac{\partial}{\partial t} \Psi(x,t) = \sqrt{- \hbar^2 c^2 \frac{\partial^2}{\partial x^2} + m^2 c^4} \Psi(x,t)
  \]
  となる。
  ここで、根号を外すために両辺を2乗すると
  \[
  \big(i \hbar \frac{\partial}{\partial t}\big)^2 \Psi = \big(-i \hbar \frac{\partial}{\partial x}\big)^2 c^2 \Psi + m^2 c^4 \Psi
  \]
  が得られる。これを上記の式に代入すると
  \[
  - \hbar^2 \frac{\partial^2}{\partial t^2} \Psi = - \hbar^2 c^2 \frac{\partial^2}{\partial x^2} \Psi + m^2 c^4 \Psi
  \]
  となり、整理すると
  \[
  \frac{1}{c^2} \frac{\partial^2}{\partial t^2} \Psi - \frac{\partial^2}{\partial x^2} \Psi + \frac{m^2 c^2}{\hbar^2} \Psi = 0
  \]
  となる。これがクライン=ゴルドン方程式である。
  ここでダランベルシアン $\Box$ を以下のように定義する
  \[
  \Box \coloneqq \frac{1}{c^2} \frac{\partial^2}{\partial t^2} - \frac{\partial^2}{\partial x^2}.
  \]
  すると、クライン=ゴルドン方程式は
  \[
  (\Box + \frac{m^2 c^2}{\hbar^2}) \Psi = 0
  \]
  と簡潔に表される。
  また、付録にて$E^2 = p^2c^2 + m^2c^4$を直接置換する方法も示す。

  \section{付録}
  \subsection{Stoneの定理}
  Stoneの定理は、強連続な1パラメータユニタリ群が必ず自己共役演算子を生成子として持つ、という定理である。具体的には、強連続な1パラメータユニタリ群 $U(t)$ に対して、自己共役演算子 $A$ が存在し、
  \[
  U(t) = e^{-i t A}
  \]
  と表される。
  \subsection{ランダウの記法}
  ランダウの記法は、関数の漸近的な振る舞いを表すための記法である。例えば、関数 $f(x)$ をテイラー展開したとき、
  \[
  \sum_{n=0}^{\infty} \frac{f^{(n)}(0)}{n!} x^n = f(0) + f'(0)x + \frac{f''(0)}{2!}x^2 + \frac{f'''(0)}{3!}x^3 + ...
  \]
  と表されるが、微小な $x$ に対しては高次の項は相対的に極めて小さくなるため、これらをまとめて無視できるほど小さいと言うことができる。このとき、$f(x)$ の高次の項をまとめて
  \[
  f(x) = O(x^n) \quad (x \to 0)
  \]
  と表す。ここで、$O(x^n)$ は「$x$ が0に近づくとき、$f(x)$ の高次の項は $x^n$ に比例するほど小さい」という意味である。
  \subsection{正準交換関係とストーン・フォン・ノイマンの定理}
  量子力学における位置演算子 $\hat{x}$ と運動量演算子 $\hat{p}$ は、ハミルトン力学における位置 $x$ と運動量 $p$ の正準交換関係
  \[
  \left\{x, p\right\} = 1
  \]
  を量子化したものであり、以下の交換関係を満たす :
  \[
  [\hat{x}, \hat{p}] = \hat{x}\hat{p} - \hat{p}\hat{x} = i\hbar I.
  \]
  ここで、$I$ は恒等作用素である。
  \subsection{クライン=ゴルドン方程式の直接置換}
  相対論的ハミルトニアンの場合、エネルギー $E$ と運動量 $p$ の関係は
  \[
  E^2 = p^2 c^2 + m^2 c^4.
  \]

  エネルギー・運動量演算子を
  \[
  \hat{E} = i \hbar \frac{\partial}{\partial t}, \quad \hat{p} = - i \hbar \frac{\partial}{\partial x}
  \]
  とすると
  \[
  \big(i \hbar \frac{\partial}{\partial t}\big)^2 \Psi = \big(-i \hbar \frac{\partial}{\partial x}\big)^2 c^2 \Psi + m^2 c^4 \Psi
  \]
  が得られる。
  これを上記の式に代入すると
  \[
  - \hbar^2 \frac{\partial^2}{\partial t^2} \Psi = - \hbar^2 c^2 \frac{\partial^2}{\partial x^2} \Psi + m^2 c^4 \Psi
  \]
  となり、整理すると
  \[
  \frac{1}{c^2} \frac{\partial^2}{\partial t^2} \Psi - \frac{\partial^2}{\partial x^2} \Psi + \frac{m^2 c^2}{\hbar^2} \Psi = 0
  \]
  となる。これがクライン=ゴルドン方程式である。

  $\LaTeX$
\end{document}
