\documentclass[aps,prd,twocolumn]{revtex4-2}

\usepackage{amsmath,amssymb}
\usepackage{graphicx}
\usepackage[colorlinks=true,
  linkcolor=blue,
  citecolor=blue,
  urlcolor=blue]{hyperref}

\newcommand{\Retr}{\mathrm{Retr}}


\begin{document}

\title{QCDのラグランジアンから格子QCDの作用への導出}

\author{しん}
\affiliation{
% 所属
}

\date{\today}

\begin{abstract}
  本稿ではQCDのラグランジアンから格子QCDの作用を導出する過程について説明する。
\end{abstract}

\maketitle

\section{lattice QCDの作用}
はじめに、QCDのラグランジアンは以下のように与えられる。
\begin{equation}
\mathcal{L}_{QCD} = \bar{\psi}(i\gamma^\mu D_\mu - m)\psi - \frac{1}{4}F_{\mu\nu}^a F^{a\mu\nu}
\end{equation}

\subsection{連続理論の作用}
\begin{equation}
S = \int d^4x \, \mathcal{L}_{QCD}
\label{eq:QCD_action}
\end{equation}

\subsection{格子の定義と積分和への変換}
格子間隔を$a$とし、格子点を
\begin{equation}
x_\mu = n_\mu a \quad (n_{0\cdots\mu} \in \mathbb{Z})
\end{equation}
で定義する(以後$\hat\mu$は$\mu$方向の単位ベクトル)。このとき、積分は以下のように和に置き換えられる。
\begin{equation}
\int d^4x \rightarrow a^4 \sum_{n}
\end{equation}
より、作用は
\begin{equation}
S \to S_{lattice} = a^4 \sum_{n} \mathcal{L}_{QCD}(n)
\end{equation}
と離散化される。

\subsection{ゲージ場:リンク変数とプラケット}
格子上ではゲージ場はリンク変数
\begin{equation}
  U_\mu(x) \in SU(3), \quad \therefore U_\mu(x)=\exp\bigl( iag\,A_\mu(x+a\frac{\hat{\mu}}{2})\bigr)
\end{equation}
で表される。
これは、格子点$x$から$x+a\hat{\mu}$へのリンクに対応する。
ここで$iagA_\mu$はSU(3)の生成子に展開される。
$A_\mu(x)$はゲージ場であり、$g$は結合定数である。
また、$(x+a\frac{\hat{\mu}}{2})$はリンクの中点を表す。
次にプラケットは、
\begin{equation}
  U_{\mu\nu}(x) = U_\mu(x) U_\nu(x+a\hat{\mu}) U_\mu^\dagger(x+a\hat{\nu}) U_\nu^\dagger(x)
\label{eq:plaquette}
\end{equation}
で定義される。プラケットとは格子上の最小の閉じたループであり、ゲージ場の曲率を表す。
これは$x \to \mu \to \mu\nu \to \nu \to x$の閉じた周回を表す。
これは、テイラー展開により
\begin{equation}
  U_{\mu\nu}(x)=\exp\bigl(i a^2 g F_{\mu\nu}(x)+\mathcal{O}(a^3)\bigr)
\end{equation}
と表され、$F_{\mu\nu}$は場の強さテンソルである。

\subsection{格子QCDの作用の導出}
格子QCDの作用は、ウィルソン作用として知られている。
導出は付録に示すが、結果は以下の通りである。
\begin{equation}
S_{lattice} = \beta \sum_{x,\mu<\nu} \left(1 - \frac{1}{3}\Retr U_{\mu\nu}(x)\right)
\end{equation}
一般に、$\beta = \frac{6}{g^2}$である。
$\mu<\nu$は各プラケットを一度だけ数えるための条件である。これを$$\mu, \nu$$と示してしまうと、各プラケットが二度数えられてしまう。

\section{lattice QCDにおけるフェルミオン}
格子QCDにおけるフェルミオンの取り扱いは、wilsonフェルミオンやstaggeredフェルミオンなど、いくつかの方法が存在する。

\begin{itemize}
  \item Wilsonフェルミオン: フェルミオンのダブリング問題を解決するために、ウィルソン項を導入する方法。
  \item Staggeredフェルミオン: フェルミオンの自由度を減らし、計算コストを削減する方法。
  \item Domain Wallフェルミオン: 5次元格子を用いて、カイラル対称性を保つ方法。
  \item Overlapフェルミオン: カイラル対称性を厳密に保つ方法。
\end{itemize}
今回はWilsonフェルミオンについて簡単に説明する。
Wilsonフェルミオンの作用は以下のように与えられる。
\begin{equation}
\begin{split}
  S_F &= a^4 \sum_{x} \bar{\psi}(x) \left( m + \frac{4r}{a} \right) \psi(x) \\
  &- \frac{a^3}{2} \sum_{x, \mu} \left[ \bar{\psi}(x) (r - \gamma_\mu) U_\mu(x) \psi(x + a\hat{\mu}) \right. \\
  &\quad + \left. \bar{\psi}(x + a\hat{\mu}) (r + \gamma_\mu) U_\mu^\dagger(x) \psi(x) \right]
\end{split}
\end{equation}
ここで、$r$はウィルソンパラメータであり、通常$r=1$と設定される。
\clearpage

\section{付録 wilson作用の導出}
ウィルソン作用の導出は以下の通りである。
まず、場の強さテンソル$F_{\mu\nu}$は
\begin{equation}
F_{\mu\nu} = \partial_\mu A_\nu - \partial_\nu A_\mu - i g [A_\mu, A_\nu]
\end{equation}
で定義される。格子上では、$F_{\mu\nu}$はプラケット$U_{\mu\nu}$式\eqref{eq:plaquette}を用いて近似される。
テイラー展開により、
\begin{equation}
U_{\mu\nu}(x) = \exp\left( i a^2 g F_{\mu\nu}(x) + \mathcal{O}(a^3) \right)
\end{equation}
が得られる。
次に、作用について解きたいため、$U_{\mu\nu}$のトレースを計算し、無次元化する。
また、$U_{\mu\nu}$はユニタリ行列であるため、複素数を含むが、複素数の作用は定義できないため、実部を取る。
\begin{equation}
\Retr U_{\mu\nu}(x) = \Retr \exp\left( i a^2 g F_{\mu\nu}(x) + \mathcal{O}(a^3) \right)
\end{equation}
これをテイラー展開すると、
\begin{equation}
\Retr \left( 1 + i a^2 g F_{\mu\nu}(x) - \frac{a^4 g^2}{2} F_{\mu\nu}^2(x) + \mathcal{O}(a^6) \right)
\end{equation}
となる。ここで、$F_{\mu\nu}$はSU(3)の生成子に展開されるため、$\Retr F_{\mu\nu} = 0$であることを利用すると1次の項が消えるため、
\begin{equation}
\Retr U_{\mu\nu}(x) = \Retr \left( 1 - \frac{a^4 g^2}{2} F_{\mu\nu}^2(x) + \mathcal{O}(a^6) \right)
\end{equation}
となる。さらに、ここでの$1$は$3 \times 3$の単位行列であるため、$\Retr 1 = 3$であることを利用すると、
\begin{equation}
\Retr U_{\mu\nu}(x) = 3 - \frac{a^4 g^2}{2} \Retr (F_{\mu\nu}^2(x)) + \mathcal{O}(a^6)
\end{equation}
が得られる。
ここで、定数項$3$は作用に寄与しないため、$\Retr U_{\mu\nu}(x)$から引く。
その時、$1$に対する偏差として表すため、
\begin{equation}
1 - \frac{1}{3} \Retr U_{\mu\nu}(x) \approx \frac{a^4 g^2}{6} \Retr (F_{\mu\nu}^2(x))
\end{equation}
と整理される。
これを作用式\eqref{eq:QCD_action}に代入すると、
\begin{equation}
S = \int d^4x \, \mathcal{L}_{QCD} \approx \frac{1}{4} \int d^4x \, F_{\mu\nu}^a F^{a\mu\nu}
\end{equation}
となる。これを格子上に離散化すると、
\begin{equation}
S_{lattice} = \beta \sum_{x,\mu<\nu} \left(1 - \frac{1}{3}\Retr U_{\mu\nu}(x)\right)
\end{equation}
が得られる。
ここで、$\beta$は以下のように定義される。
$\mathcal{L}_{\text{gauge}} = \frac{1}{4} F_{\mu\nu}^a F^{a\mu\nu}$の係数を一致させるためである。
実際に
\begin{equation}
\beta \cdot \frac{a^4 g^2}{6} = \frac{1}{4} a^4
\end{equation}
であるため、
\begin{equation}
\beta = \frac{6}{g^2}
\end{equation}
が得られる。


\end{document}